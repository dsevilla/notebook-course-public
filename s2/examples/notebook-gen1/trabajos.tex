\input header.tex

%-----------------------------------------------------------------------------
%       TITLE PAGE
%-----------------------------------------------------------------------------

\lstset{
  numbers=none
  }


\title[BDGE - Trabajos]%
{Bases de Datos a Gran Escala\\%
  Trabajos}

\author{Diego Sevilla Ruiz}
\institute[UMU]
{
Dpto. Ingeniería y Tecnología de Computadores\\
Facultad de Informática\\
Universidad de Murcia\\
\medskip
\href{mailto:dsevilla@um.es}{\textit{dsevilla@um.es}}
}
\date{2019}

\makeatletter
\patchcmd{\beamer@sectionintoc}{\vskip1.5em}{\vskip1em}{}{}
\makeatother

\begin{document}


\begin{frame}
  \titlepage
\end{frame}


% \begin{frame}
% \frametitle{Índice}
% \tableofcontents[hideallsubsections]
% \end{frame}

\section{Instrucciones para realizar los trabajos}

\begin{frame}[fragile,allowframebreaks]
  \frametitle{Instrucciones para la realización de los trabajos}
\begin{itemize}
\item En las siguientes transparencias se incluyen los trabajos propuestos
  para la asignatura
\item Cada trabajo incluye el estudio de una base de datos vista en clase y
  la importación de los datos de Stackoverflow
\item El trabajo será \underline{\bf INDIVIDUAL}
\item Más de un alumno podrá elegir el mismo trabajo, pero como máximo un
  trabajo podrá ser elegido por dos alumnos
\item Se entregará una memoria (y código, si se ha generado) en una tarea
  abierta a tal fin
\item La fecha de entrega será el {\bf día del examen de la asignatura}
\item Para apuntarse a un trabajo el alumno tendrá que:

  \framebreak

  \begin{enumerate}
  \item Conectarse al servidor (usuario: alumno, pass: bdge2019)
\begin{lstlisting}[language=bash]
$ mysql -hneuromancer.inf.um.es -P3307 \
    -ualumno -pbdge2019 \
    --default-char-set=utf8 \
    --protocol=tcp trabajos
  \end{lstlisting}
%
O bien si se usan los contenedores instalados:
%
\begin{lstlisting}[language=bash]
$ docker run --rm -ti mysql mysql \
    -hneuromancer.inf.um.es -ualumno \
    -pbdge2019 --protocol=tcp -P3307 \
    --default-character-set=utf8 trabajos
\end{lstlisting}

  \item Añadir una entrada a la tabla {\tt asignacion\_trabajos} con su
    dni, nombre y id del trabajo elegido
  \item Se pueden listar qué trabajos hay y cuáles quedan libres: (no se
    muestra el campo {\tt spec}, con la especificación del trabajo)
    %
    \begin{lstlisting}[language=sql,basicstyle=\tiny\tt]
mysql> select id,titulo from trabajos;
+-----+-----------------------+
| id  | titulo                |
+-----+-----------------------+
| T01 | Open TSDB             |
| T02 | Apache Cassandra      |
+-----+-----------------------+
\end{lstlisting}
%
\begin{lstlisting}[language=sql,basicstyle=\tiny\tt]
mysql> select * from asignados;
+-----+-----------------------+------------+
| id  | titulo                | nasignados |
+-----+-----------------------+------------+
| T01 | Open TSDB             |          0 |
| T02 | Apache Cassandra      |          0 |
+-----+-----------------------+------------+
\end{lstlisting}
  \end{enumerate}

\framebreak

\item Las tablas disponibles:
%
\begin{lstlisting}[language=sql]
mysql> show tables;
+---------------------+
| Tables_in_trabajos  |
+---------------------+
| asignacion_trabajos |
| asignados           |
| trabajos            |
+---------------------+
\end{lstlisting}

\item El código de creación de la base de datos es un ejemplo de creación
  de usuarios y de permisos, y puede verse en el {\tt git} de la asignatura
  aquí:
  \url{https://github.com/dsevilla/bdge/blob/19-20/addendum/trabajos/creatrabajos.sql}

\end{itemize}


\end{frame}

\section{Trabajos para la asignatura BDGE}

\begin{frame}
  \frametitle{T01. Open TSDB}
  \begin{itemize}
  \item Pasos de instalación de la base de datos
  \item Descripción de la base de datos, modo de funcionamiento,
    posibilidades de modelado de datos y características (si permite
    transacciones, framework de procesamiento map/reduce, replicación
    multiservidor, etc.)
  \item Mostrar cómo importar los datos de Stackoverflow
  \item Mostrar cómo redistribuir los datos de Stackoverflow de forma
    óptima (uso de series de datos, como si, por ejemplo, comentarios,
    posts, etc. se ejecutaran en un stream)
  \item Mostrar cómo se realizarían las consultas RQ1 a RQ4 de los
    artículos vistos en la sesión~2
  \item Realizar pruebas de eficiencia comparada con alguna de las bases de
    datos vistas en la asignatura
  \item \url{http://opentsdb.net/}
  \end{itemize}
\end{frame}


\begin{frame}
  \frametitle{T02. Apache Cassandra}
  \begin{itemize}
  \item Pasos de instalación de la base de datos
  \item Descripción de la base de datos, modo de funcionamiento,
    posibilidades de modelado de datos y características (si permite
    transacciones, framework de procesamiento map/reduce, replicación
    multiservidor, lenguaje de consultas CQL, etc.)
  \item Mostrar cómo importar los datos de Stackoverflow
  \item Mostrar cómo redistribuir los datos de Stackoverflow de forma
    óptima (uso de agregación siguiendo el modelo de documentos)
  \item Mostrar cómo se realizarían las consultas RQ1 a RQ4 de los
    artículos vistos en la sesión~2
  \item Realizar pruebas de eficiencia comparada con alguna de las bases de
    datos vistas en la asignatura
  \end{itemize}
\end{frame}

\end{document}

%%% Local variables:
%%% mode: LaTeX
%%% TeX-master: "header.tex"
%%% ispell-local-dictionary: "spanish"
%%% fill-column: 75
%%% TeX-parse-self: t
%%% TeX-auto-save: t
%%% End:
%%% vim: expandtab shiftwidth=2 tabstop=2 ai spelllang=es_ES spell tw=80
