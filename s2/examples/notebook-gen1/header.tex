\PassOptionsToPackage{unicode}{hyperref}
\PassOptionsToPackage{naturalnames}{hyperref}

\documentclass[12pt]{beamer}

\mode<presentation> {

% The Beamer class comes with a number of default slide themes
% which change the colors and layouts of slides. Below this is a list
% of all the themes, uncomment each in turn to see what they look like.

%\usetheme{default}
%\usetheme{AnnArbor}
%\usetheme{Antibes}
%\usetheme{Bergen}
%\usetheme{Berkeley}
%\usetheme{Berlin}
%\usetheme{Boadilla}
%\usetheme{CambridgeUS}
%\usetheme{Copenhagen}
%\usetheme{Darmstadt}
%\usetheme{Dresden}
%\usetheme{Frankfurt}
%\usetheme{Goettingen}
%\usetheme{Hannover}
%\usetheme{Ilmenau}
%\usetheme{JuanLesPins}
%\usetheme{Luebeck}
\usetheme{Madrid}
%\usetheme{Malmoe}
%\usetheme{Marburg}
%\usetheme{Montpellier}
%\usetheme{PaloAlto}
%\usetheme{Pittsburgh}
%\usetheme{Rochester}
%\usetheme{Singapore}
%\usetheme{Szeged}
%\usetheme{Warsaw}

% As well as themes, the Beamer class has a number of color themes
% for any slide theme. Uncomment each of these in turn to see how it
% changes the colors of your current slide theme.

%\usecolortheme{albatross}
%\usecolortheme{beaver}
%\usecolortheme{beetle}
\usecolortheme{crane}
%\usecolortheme{dolphin}
%\usecolortheme{dove}
%\usecolortheme{fly}
%\usecolortheme{lily}
%\usecolortheme{orchid}
%\usecolortheme{rose}
%\usecolortheme{seagull}
%\usecolortheme{seahorse}
%\usecolortheme{whale}
%\usecolortheme{wolverine}

%\setbeamertemplate{footline} % To remove the footer line in all slides uncomment this line
%\setbeamertemplate{footline}[page number] % To replace the footer line in all slides with a simple slide count uncomment this line

\setbeamertemplate{navigation symbols}{} % To remove the navigation symbols from the bottom of all slides uncomment this line
\setbeamertemplate{frametitle continuation}[from second][\insertcontinuationtext]
}

\usepackage{graphicx} % Allows including images
\usepackage{booktabs} % Allows the use of \toprule, \midrule and \bottomrule in tables
\usepackage[T1]{fontenc}
\usepackage[spanish]{babel}
\usepackage[utf8]{inputenc}
\usepackage{listings}
\usepackage{tikz}
\usetikzlibrary{positioning,backgrounds}
\usetikzlibrary{decorations.pathmorphing}
\usepackage{iwona}
\usepackage{marvosym}
%\usepackage{cfr-lm}
%\usepackage{pifont}
%\usepackage{keystroke}
\usepackage{etoolbox}

%
% Listados de código
%
\lstset{%
basicstyle=\ttfamily\footnotesize,
commentstyle=\color{gray}\itshape\ttfamily,
keywordstyle=\color{blue!80}\bfseries\ttfamily,
stringstyle = \color{gray},
showstringspaces=false,
frame=tblr, % single, tb, ltrb % boxed listings, en mayusculas = doble linea
framerule=0pt,
tabsize=4, % tabulador = 2 espacios
captionpos=b,
breaklines=true,
%backgroundcolor=\color{white},
numbers=left, numberstyle=\tiny, stepnumber=2, numbersep=10pt,
xleftmargin=0.02\textwidth,
xrightmargin=0.02\textwidth,
language=java, % Por defecto
literate={«}{{\guillemotleft}}1
           {»}{{\guillemotright}}1
           {é}{{\'e}}1
           {í}{{\'i}}1
           {ó}{{\'o}}1
           {ú}{{\'u}}1
           {á}{{\'a}}1
           {ñ}{{\~n}}1
           {Ñ}{{\~N}}1
           {¿}{{?`}}1
}


%% Macros comunes
\newcommand{\hide}[1]{}
\newcommand{\ra}{$\rightarrow${}~{}}


% \end{document}

%%% Local variables:
%%% mode: LaTeX
%%% TeX-master: t
%%% ispell-local-dictionary: "spanish"
%%% fill-column: 75
%%% TeX-parse-self: t
%%% TeX-auto-save: t
%%% End:
%%% vim: expandtab shiftwidth=2 tabstop=2
